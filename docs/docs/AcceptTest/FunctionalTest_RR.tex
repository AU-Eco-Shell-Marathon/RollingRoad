\section{Test of functional requirements}
Text

\begin{table}[h!]
	\centering
	\label{my-label}
	\begin{tabular}{|p{1.5 cm}|p{2.1 cm}|p{2.1 cm}|p{2.1 cm}|p{2.1 cm}|p{2.1 cm}|}
		\hline
		\textbf{Req. \#} & \textbf{Description} & \textbf{Test procedure} & 
		\textbf{Expected result} & \textbf{Actual result} & \textbf{Accept/ Comment} \\ \hline
		RR\_F1 
		& Must be able to communicate with a PC and an associated GUI.
		& The Control Unit is connected to a PC.
		& The GUI displays a message which validates the connection.
		&
		& \\ \hline
		RR\_F2
		& The user must be able to regulate the load in the system via the GUI.
		& The user sets a chosen load (measured in N$\cdot$m) and applies a equal amount of torque to the Roll.
		& The Roll stops.
		& 
		& \\ \hline
		RR\_F3 
		& Must be able to display the collected results on the GUI.
		& The Wheel is spun in order to create a certain amount of torque.
		& The results are displayed on the GUI and graphs the data over time.
		& 
		& \\ \hline
		RR\_F4 
		& Must be able to perform an automatic shutdown when a current of 21 A is measured in the system.
		& A current of 21 A is applied to the circuit.
		& The GUI displays an error-message and the µC shuts down.
		&
		& \\ \hline
		RR\_F5 
		& Must be able to measure the power from the battery and the torque given to the generator. 
		& 
		&
		&
		& \\ \hline
	\end{tabular}
	\caption{Test of the functional requirements for Rolling Road}
\end{table}